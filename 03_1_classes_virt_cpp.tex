\section{Abgeleiteten Klassen}
Dieser Abschnitt zeigt einen vollständigen Beispiel einer Vererbung in C++.
In einem Unternehmen werden von den gewöhnlichen Angestellten nur der Vorname und Nachname gespeichert.
Es ist möglich den Vornamen und Namen ausgeben zu lassen.

\begin{code}
	\caption{C++ Beispiel: Employee Klasse}
	\label{code:virtcpp:employee}
	\inputminted{cpp}{code/employees_virt_cpp/employee.hpp}
	%\inputminted{cpp}{code/employees_virt_cpp/employee.cpp}
\end{code}

Ein Manager ist ein Angestellter mit einem bestimmten Level.
Ihm kann eine Gruppe bis maximal von 10 Angestellten zugeordnet werden.
Es ist möglich alle diese Information über einen Manager anzeigen zu lassen.

\begin{code}
	\caption{C++ Beispiel: Manager Klasse}
	\label{code:virtcpp:manager}
	\inputminted{cpp}{code/employees_virt_cpp/manager.hpp}
	%\inputminted{cpp}{code/employees_virt_cpp/manager.cpp}
\end{code}

Dieser Zusammenhang ist abgebildet durch die Klassen \mintinline{cpp}{Employee} und \mintinline{cpp}{Manager}.


Eine mögliche Nutzung ist in \Cref{code:virtcpp:usage} dargestellt.
Wir haben in den Konstruktor, Destruktor und Memberfunktionen Debugging-Code eingeschleusst um die Aufrufreihenfolge zu verfolgen.
Als Kommentar zur jeder Ausgabe steht auch der Vorname und der Nachname der Angestellten, um die Aufrufe zuzuordnen zu können.
In der Ausgabe kann man sehen, dass beim Erzeugen des Objektes stets ein passender Konstruktor aufgerufen wird.
Am Ende werden die Objekte nach dem LIFO Prizip automatisch zerstört, weil der Gültigkeitsbereich der Objekte, der sich auf die \mintinline{cpp}{main} Funktion beschränkt, verlassen wird.
Zwischen den Konstruktor- und Destruktoraufrufen sehen wir die Aufrufend er Memberfunktionen und die Ausgabe von \mintinline{cpp}{printf}.

\begin{code}
	\caption{C++ Beispiel: Nutzung}
	\label{code:virtcpp:usage}
	\inputminted{cpp}{code/employees_virt_cpp/main.cpp}
\end{code}

\begin{code}
	\caption{C++ Beispiel: Ausgabe}
	\label{code:virtcpp:output}
	\inputminted[bgcolor=white]{text}{code/employees_virt_cpp/output.txt}
\end{code}

Dieses Verhalten wollten wir in C nachbilden.
