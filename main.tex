\documentclass[a4paper]{article}

\usepackage[T1]{fontenc}
%\usepackage[utf8]{inputenc}

\usepackage[a4paper,left=2cm,right=2cm,top=2cm,bottom=2cm]{geometry}

\usepackage{silence}
\WarningFilter{latex}{Command}
\WarningFilter{fvextra}{csquotes should be loaded after fvextra}

\usepackage{lmodern}
\usepackage{csquotes}
%\usepackage[style=numeric,backend=biber]{biblatex}
\usepackage[style=numeric]{biblatex}
\addbibresource{literatur.bib}

%\usepackage[english]{babel}
\usepackage[ngerman]{babel}

\usepackage{xcolor}
\usepackage{caption}
\usepackage[newfloat,cache=false]{minted}

\makeatletter
\newcommand{\currentfontsize}{\fontsize{\f@size}{\f@baselineskip}\selectfont}
\makeatother
\setmintedinline{fontsize=\currentfontsize}

%\DeclareCaptionFormat{myformat}{\fontsize{5}{6}\selectfont#1#2#3}
%\captionsetup{format=myformat}

%\newenvironment{code}{\captionsetup{type=listing, format=myformat}}{}
\newenvironment{code}{\captionsetup{type=listing}}{}
\SetupFloatingEnvironment{listing}{name=Source Code}

\definecolor{bg}{HTML}{282828}
\setminted{style=monokai, fontsize=\small, bgcolor=bg, linenos}

\usepackage[most]{tcolorbox}

\newtcolorbox{note}[1][]{%
  enhanced jigsaw, % better frame drawing
  borderline west={2pt}{0pt}{red}, % straight vertical line at the left edge
  sharp corners, % No rounded corners
  boxrule=0pt, % no real frame,
  fonttitle={\large\bfseries},
  coltitle={black},  % Black colour for title
  title={Note:\ },  % Fixed title
  attach title to upper, % Move the title into the box
  #1
}


\newtcolorbox{missing}[1]{
 % colbacktitle=titlebg, 
	%coltitle=titlefg, 
	%colback=bodybg,
 % colupper=bodyfg,
	breakable,
	fonttitle=\bfseries,
	title={#1}
}

\usepackage{hyperref}
\usepackage{cleveref}
\usepackage{microtype}
\usepackage{textcomp}
\usepackage{units}
\usepackage{tikz}
\usetikzlibrary{arrows}

\renewcommand\mkbibacro[1]{{\footnotesize\MakeUppercase{#1}}}

\graphicspath{
	{./pictures/}
}


\title{Objektorientierte Programmierung mit C}
\author{Eugen Betke}
%\institute{Arbeitsbereich Wissenschaftliches Rechnen\\Fachbereich Informatik\\Fakultät für Mathematik, Informatik und Naturwissenschaften\\Universität Hamburg}
\date{2019-01-14}


\begin{document}
\maketitle

\tableofcontents

\section{Einleitung}
Obwohl C keine objektorientierte Programmierung unterstützt, kann man mit der Sprache den Konzept der objektorientierten Programmierung umsetzen.
Das erfordert allerdings ein sehr viel Disziplin, weil C-Sprache den Programmierer kaum dabei unterstützt.
Sie eschwert es sogar, indem sie viele Umsetzungsmöglichkeiten bietet.
In dieser Ausarbeitung behandeln nur eine Möglichkeit von Vielen objektorientiert zu Programmieren.
Unsere Absicht ist möglichst nicht über den C-Standard hinausgehen, d.h. ohne Verwendung von exotischen Kompilererweiterungen und mit minimalen Einsatz von Präprozessor.


\subsection{Debugging}
Um die Funktionsweise zu erläutern schleusen wir Debugging-Code in die Programme.
Es handelt sich dabei um die Ausgabe des Funktionsnamen, der Zeile und Kommentar.
Das Macro ist dargestellt in \Cref{code:debug}

\begin{code}
	\caption{Debugging Macro}
	\label{code:debug}
	\inputminted{c}{code/examples/employees_virt/debug.h}
\end{code}

\subsection{Kompilierung}
Die anonymen Strukturen haben sich als esseziell für die virtuellen Strukturen erwiesen.
Um sie zu aktivieren, verwenden wir bei der Kompilierung den folgenden Kompiler-Flag.
\begin{code}
	\begin{minted}{bash}
gcc -fms-extensions
	\end{minted}
\end{code}


\section{Einfache Klassen}
Eine Klasse definiert einen Bauplan für einen neuen Datentypen.
Allerdings, bietet uns die C-Programmiersprache keine echten Klassen an.
Das Nächste, was Klassen nahe kommt sind Strukturen.
Sie spielen deshalb eine zentrale Rolle bei der Umsetzung des OOP Konzeptes.

\subsection{Klassendeklaration}

In C werden, bis auf einige Ausnahmen, die Variablen, insbesondere Strukturen, nicht automatisch initialisiert.
Nach Instanziierung der Strukturen werden wir deswegen üblicherweise zufällige Werte in den Membervariablen vorfinden.
Es ist die Aufgabe der Programmierer die Instanzen in einen Konsistenten zustand zu bringen.
Desweiteren, werden wir in statt Strukturen, von den Klassen sprechen.
Die Instanzen der Strukturen nennen wir Objekte.

Bevor wir mit dem Konzept fortfahren, wollen wir einen generellen Blick auf die OOP in den anderen Programmiersprachen werfen.
Die nativen OOP Programmiersparachen wie C++ und Java erzeugen automatisch eine Reihe von Default-Funktionen, die bestimmte Operationen auf Objekten ausführen, wenn der Programmier sie selber nicht erzeugt.
Zu einem sind es Konstruktoren, die dafür zuständig sind den konsisteten Zustand von Objekten sicherzustellen.
Jede Klasse benötigt mindestens einen Konstruktor.
Abhängig von dem Funktionalität, kann eine Klasse mehrere Konstruktoren haben.
Weitere wichtige Funktionen sind Kopierkonstruktor, Zuweisungsoperator und Destruktor.
Um sauber das OOP-Konzept umzusetzen, benötigen wir auch diese Funktionen.
Da C-Programmiersprache sie nicht automatisch erzeugt, muss der Programmier sich darum kümmern.

\begin{itemize}
 \item Kopierkonstruktor
	 \begin{itemize}
		 \item Ein Kopierkonstruktor wird auf uninitialisierte Objekte angewendet. Er weist eine Kopie des Quellobjektes einem uninitialisierten Zielobjekt zu.
\begin{minted}{c}
someclass_t target;
someclass_t source;
someclass_constructor(&target, /* init params */)
someclass_copy(&target, &source);
/* ... */
\end{minted}
	 \end{itemize}
 \item Zuweisungsoperator
	 \begin{itemize}
		 \item Ein Zuweisungsoperator wird auf bereits initializierte Objekte angewendet. Das Zielobjekt wird zuerst bereinigt. Danach wird eine Kopie aus dem Quellobjekt erzeugt und dem Zielobjekt zugewiesen. So entstehen keine Speicherlecks.
\begin{minted}{c}
someclass_t target;
someclass_t source;
someclass_constructor(&target, /* init params */)
someclass_constructor(&source, /* init params */)
someclass_assign(&target, &source);
/* ... */
\end{minted}
	 \end{itemize}
 \item Destruktor
	 \begin{itemize}
		 \item Ruf Destruktoren von allen Membervariablen des Objektes.
	 \end{itemize}
\end{itemize}

Der Destruktor ist ein besonders wichtiger Bestandteil von OOP.
Insbesondere, weil auf dem Heap allozierter Speicher nach der Verwendung wieder freigegben werden muss, um Speicherlecks zu vermeiden.
Für diesen Zwecke muss zu einem geeigneten Zeitpunkt der Destruktor aufgerufen werden.
In C muss Destruktor manuell aufgerufen werden, da C uns keine Hilfmittel für die Automatisierung zur Verfügung stellt, z.B. kein Garbage-Kollektor oder Aufruf vom Destruktor beim Verlassen des Gültigekeitsbereiches des Objektes.

In unseren fiktiven Beispiel wollen wir den Vornamen, den Namen und den Job, der die Angestellten einer Firma ausüben speichern.
In \Cref{code:plain:header} beinhaltet die Strukture \mintinline{c}{employee_t} die Membervariablen.
Weiter unten werden der Konstruktor, Destruktor, Kopierkonstruktor und der Zuweisungsoperator deklariert.
Zusätzlich deklarieren wir eine \mintinline{c}{print} Memberfunktion.


\begin{code}
	\caption{Einfache Klassen: employees.h}
	\label{code:plain:header}
	\inputminted{C}{code/examples/employees_plain/employee.h}
\end{code}



\subsection{Klassenfunktionen}
In C bietet es sich an die Implementierung der Funktionen in separate Source-Dateien auszulagern.
Wie die Implementierung aussehen kann zeigt \Cref{code:plain:source}.

\begin{code}
	\caption{Einfache Klassen: employees.c}
	\label{code:plain:source}
	\inputminted{C}{code/examples/employees_plain/employee.c}
\end{code}


\subsection{Nutzung}
Bei der Benutzung sollte man immer im Erinnerung behalten, dass alle Funktionen, mindestens ein Konstruktor und Destruktor manuell aufrufen werden müssen.
\Cref{code:plain:usage} zeigt anhand von Beispielen wie der Konstruktor, Kopierkonstruktor, Zuweisungsoperator und der Destruktor angewendet werden können.

\begin{code}
	\caption{Einfache Klassen: main.c}
	\label{code:plain:usage}
	\inputminted{C}{code/examples/employees_plain/main.c}
\end{code}

%\begin{code}
%  \caption{Einfache Klassen: Ausgabe}
%  \label{code:plain:output}
%  \inputminted[bgcolor=white]{text}{code/examples/employees_plain/output.txt}
%\end{code}

\subsection{Einschränkungen}
\subsubsection{Zugriffskontrolle}
C-Standard bietet kein Hilfsmittel, mit den mit den man den Zugriff auf die Klassenvariablen und Klassenfunktionen beschränken könnte. 
Alle Variablen und Methoden sind quasi \mintinline{C}|public|. 
Nachbildung von \mintinline{C}|protected| \mintinline{C}|private| nicht ohne Weiteres möglich.

\subsubsection{Konstruktor und Destruktor}
Wie bereits diskutiert, ist es leider so, dass beim Erzeugen eines Objektes der passende Konstruktor nicht automatisch aufgerufen wird, wie z.B. bei C++ oder Java.
Darum muss sich der Programmier explizit kümmern.
Das erfordert zwar viel Disziplin, bedeutet für uns keine Nachteile in der Funktionalität.
Bei den Destruktoren müssen wir leider Abstriche machen.
C bietet uns weder eine Garbage-Kollektor noch Ruf den Destrutkor automatisch auf, wenn das Programm den Scope verlässt.
Konkret für uns heißt es wieder viel Disziplin, aber auch das wir bestimmte Programmierkonzepte wie "RAII" nicht umsetzen können.

%\begin{note}
%  Das geht mit der clean() Erweiterung. Das ist aber keine Standard-C, somit wird diese Möglichkeit hier nicht behandelt.
%\end{note}

\subsection{C++ als Vorbild}
Dieser Abschnitt zeigt einen Beispiel einer Vererbung in C++.
In einem Unternehmen werden von den gewöhnlichen Angestellten nur der Vorname und Nachname gespeichert.
Es ist möglich den Vornamen und Namen ausgeben zu lassen.

\begin{code}
	\caption{C++ Beispiel: \mintinline{cpp}{Employee} Klasse}
	\label{code:virtcpp:employee}
	\inputminted{cpp}{code/examples/employees_virt_cpp/employee.hpp}
	%\inputminted{cpp}{code/employees_virt_cpp/employee.cpp}
\end{code}

Ein Manager ist ein Angestellter mit einem bestimmten Level.
Ihm kann eine Gruppe mit einer bestimmter Anzahl der Angestellten zugeordnet werden.
Es ist möglich alle diese Information über einen Manager anzeigen zu lassen.

\begin{code}
	\caption{C++ Beispiel: \mintinline{cpp}{Manager} Klasse}
	\label{code:virtcpp:manager}
	\inputminted{cpp}{code/examples/employees_virt_cpp/manager.hpp}
	%\inputminted{cpp}{code/employees_virt_cpp/manager.cpp}
\end{code}

Dieser Zusammenhang ist abgebildet durch die Klassen \mintinline{cpp}{Employee} und \mintinline{cpp}{Manager}.

Eine mögliche Nutzung ist in \Cref{code:virtcpp:usage} dargestellt.
Wir haben in den Konstruktor, Destruktor und Memberfunktionen Debugging-Code eingeschleusst um die Aufrufreihenfolge zu verfolgen.
Als Kommentar zur jeder Ausgabe steht auch der Vorname und der Nachname der Angestellten, um die Aufrufe zuzuordnen zu können.
In der Ausgabe kann man sehen, dass beim Erzeugen des Objektes stets ein passender Konstruktor aufgerufen wird.
Am Ende werden die Objekte nach dem LIFO Prizip automatisch zerstört, weil der Gültigkeitsbereich der Objekte, der sich auf die \mintinline{cpp}{main} Funktion beschränkt, verlassen wird.
Zwischen den Konstruktor- und Destruktoraufrufen sehen wir die Aufrufe der Memberfunktionen und die Ausgabe von \mintinline{cpp}{printf}.

\begin{code}
	\caption{C++ Beispiel: Nutzung}
	\label{code:virtcpp:usage}
	\inputminted{cpp}{code/examples/employees_virt_cpp/main.cpp}
\end{code}

\begin{code}
	\caption{C++ Beispiel: Ausgabe}
	\label{code:virtcpp:output}
	\inputminted[bgcolor=white]{text}{code/examples/employees_virt_cpp/output.txt}
\end{code}

Dieses Verhalten wollten wir in C nachbilden.

\section{Abgeleiteten Klassen}
Die Grundidee der Vererbung ist die Wiederverwendung vom Code.
Dadurch wird auch die Wartung der Programme erheblich vereinfacht.
Nach der Vererbung erhält die abgeleitete Klasse die Funktionalität der Basisklasse.

Den Preis, den wir dafür bezahlen, ist typischerweise den Speicher für ein Zeiger auf eine virtuelle Tabelle (8 bytes in der 64bit-Architektur).
Das ist aber eine Implementierungssache und der Preis kann bei anderen Implementierungen ganz anders sein.


\begin{code}
	\inputminted{cpp}{code/employees_virt_cpp/main.cpp}
	\inputminted[bgcolor=white]{text}{code/employees_virt_cpp/output.txt}
\end{code}

\begin{code}
	\caption{main}
	\inputminted{c}{code/employees_virt/main.c}
	\inputminted[bgcolor=white]{text}{code/employees_virt/output.txt}
\end{code}

%\subsection{Nameskonflikte}

Bevor wir die Vererbung implementieren, müssen wir es die Entscheidung treffen wie wir mit den Namenkonflikten umgehen.
Wir haben die Wahl zwischen einer benannten oder anonymen Basisklasse.
Bei der ersteren ist der Zugriff auf die gleichbenannte Variable in der Basisklasse immer gewährleistet.
Allerdings muss der Zugriff immer über die Basisklasse gehen.
Bei der zweiten MÖglichkeit erlaubt uns der C-Standard keine anonyme Struktur, wenn eine gleichnamige Variable bereits in der Basisklasse vorhanden ist.
Das schränkt zwar in der Wahl der Variablenname etwas ein, aber der Zugriff auf die Variablen in der Basisklasse wird kürzer.
Wir werden die erste Möglichkeit verwenden, da sie uns weniger einschränkt.

\noindent\begin{minipage}{.45\textwidth}
\begin{minted}{C}
struct A {
 int p;
};

struct B {
 A super;
 int q;
 int p; // Kein Namenskonflikt
};

B b;
b.super.p = 5;
b.q = 6;
b.p = 7;
\end{minted}
\end{minipage}\hfill
\begin{minipage}{.45\textwidth}
\begin{minted}{C}
struct A {
 int p;
};

struct B {
 A;
 int q;
 // int p; 
 // Name bereits vergeben
};

B b;
b.p = 5;
b.q = 6;
\end{minted}
\end{minipage}







%\begin{frame}[fragile]{Beispiel}
%\begin{columns}
%\column{0.65\textwidth}


\tikzset{
 treenode/.style = {align=center, inner sep=4pt, text centered, font=\sffamily},
 arn_n/.style = {treenode, rectangle, font=\sffamily}
}


\begin{figure}[h]
	\centering
\begin{tikzpicture}[<-,>=stealth',level/.style={sibling distance = 5.5cm/#1,
 level distance = 1cm}] 
 \node [arn_n] {\mintinline{C}{shape_t}}
	 child{ node [arn_n] {\mintinline{C}{line_t}} 
		 child{ node [arn_n] {\mintinline{C}{arrow_t}}}
	 }
	 child{ node [arn_n] {\mintinline{C}{cirle_t}}
		 child{ node [arn_n] {\mintinline{C}{arc_t}}}
		 child{ node [arn_n] {\mintinline{C}{angle_t}}}
	 }
; 
\end{tikzpicture}
\caption{Klassen Hierarchie}
\label{fig:hierarchy}
\end{figure}

%  \begin{figure}
%    \begin{center}
%      \includegraphics[width=0.9\textwidth]{type_sizes.png}
%    \end{center}
%    \caption{Beispielgrafik}
%  \end{figure}

\inputminted{C}{code/employees/employee.h}


%\end{frame}


%\begin{frame}[fragile]{Anonyme Strukturen}

\begin{code}
\caption{Basisklasse}
\begin{minted}{C}
struct base_t { int a; };
\end{minted}
\end{code}

\begin{code}
\caption{Anonyme Struktur}
\begin{minted}{C}
struct derived_t {
 struct base_t;
 int b;
};
\end{minted}
\end{code}

\begin{code}
\caption{Zugriff}
\begin{minted}{C}
struct derived_t d;
d.a = 5;
d.b = 6;
\end{minted}
\end{code}

\begin{code}
\caption{Beispielquelltext}
\begin{minted}{C}
struct derived_t {
 struct base_t this;
 int b;
};
\end{minted}
\end{code}

\begin{code}
\caption{Beispielquelltext}
\begin{minted}{C}
struct derived_t d;
d.this.a = 5;
d.b = 6;
\end{minted}
\end{code}






\section{Abstrakte Klassen}
\subsection{Virtuelle Funktionen}
%%https://www.embedded.com/electronics-blogs/programming-pointers/4391967/Virtual-functions-in-C
%%http://vgcoding.blogspot.com/2014/03/virtual-functions-in-c.html

%\begin{frame}[fragile]{Konzept der virtuellen Funktionen}
%  %Quelle: \cite{virtfunc2012}
%\end{frame}

%%\section{Grenzen}
%%\begin{frame}[fragile]{Eigenschaften}
%  %\begin{itemize}
%    %\item Konvetionen
%      %\begin{itemize}
%        %\item Funktionensname haben den Praefix der Struktur
%      %\end{itemize}
%  %\end{itemize}
%%\end{frame}





%%\section{Zusammenfassung}
%%\subsection*{}

%%\begin{frame}
%%  \frametitle{Zusammenfassung}
%%\end{frame}

%\section{Literatur}
%\subsection*{}

%\begin{frame}
%  %\frametitle{Literatur}
%  %\printbibliography
%\end{frame}

\end{document}
