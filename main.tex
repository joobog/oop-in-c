\documentclass[a4paper]{article}

\usepackage[T1]{fontenc}
%\usepackage[utf8]{inputenc}

\usepackage[a4paper,left=2cm,right=2cm,top=2cm,bottom=2cm]{geometry}

\usepackage{silence}
\WarningFilter{latex}{Command}
\WarningFilter{fvextra}{csquotes should be loaded after fvextra}

\usepackage{lmodern}
\usepackage{csquotes}
%\usepackage[style=numeric,backend=biber]{biblatex}
\usepackage[style=numeric]{biblatex}
\addbibresource{literatur.bib}

%\usepackage[english]{babel}
\usepackage[ngerman]{babel}

\usepackage{xcolor}
\usepackage{caption}
%\usepackage[newfloat,cache=false]{minted}
\usepackage[newfloat]{minted}

\makeatletter
\newcommand{\currentfontsize}{\fontsize{\f@size}{\f@baselineskip}\selectfont}
\makeatother
\setmintedinline{fontsize=\currentfontsize}

%\DeclareCaptionFormat{myformat}{\fontsize{5}{6}\selectfont#1#2#3}
%\captionsetup{format=myformat}

%\newenvironment{code}{\captionsetup{type=listing, format=myformat}}{}
\newenvironment{code}{\captionsetup{type=listing}}{}
\SetupFloatingEnvironment{listing}{name=Source Code}

%\definecolor{bg}{HTML}{282828}
\definecolor{bg}{HTML}{EAEAEA}
\setminted{style=default, fontsize=\small, bgcolor=bg, linenos, tabsize=2}

\usepackage[most]{tcolorbox}

\newtcolorbox{note}[1][]{%
  enhanced jigsaw, % better frame drawing
  borderline west={2pt}{0pt}{red}, % straight vertical line at the left edge
  sharp corners, % No rounded corners
  boxrule=0pt, % no real frame,
  fonttitle={\large\bfseries},
  coltitle={black},  % Black colour for title
  title={Note:\ },  % Fixed title
  attach title to upper, % Move the title into the box
  #1
}


\newtcolorbox{missing}[1]{
 % colbacktitle=titlebg, 
	%coltitle=titlefg, 
	%colback=bodybg,
 % colupper=bodyfg,
	breakable,
	fonttitle=\bfseries,
	title={#1}
}

\usepackage{hyperref}
\usepackage{cleveref}
\usepackage{microtype}
\usepackage{textcomp}
\usepackage{units}
\usepackage{tikz}
\usetikzlibrary{arrows}

\renewcommand\mkbibacro[1]{{\footnotesize\MakeUppercase{#1}}}

\graphicspath{
	{./pictures/}
}


\title{Objektorientierte Programmierung mit C}
\author{Eugen Betke}
%\institute{Arbeitsbereich Wissenschaftliches Rechnen\\Fachbereich Informatik\\Fakultät für Mathematik, Informatik und Naturwissenschaften\\Universität Hamburg}
\date{2019-01-14}

\usepackage{titlesec}
\newcommand{\sectionbreak}{\clearpage}

\begin{document}
\maketitle

\tableofcontents

\section{Einleitung}
Obwohl C keine Objekt-orientiert Programmierung unterstützt kann man die Sprache sehr wohl objekt-orientiert programmieren.
Das erfordert allerdings etwas Disziplin, weil C uns kaum unterstützt.
Diese Ausarbeitung zeigt nur eine Möglichkeit objekt-orientiert zu Programmieren von Vielen.
Hier wollen wir nur so weit gehen, wie es mit dem C-Standard möglich ist, d.h. ohne Verwendung von Kompilererweiterungen und ohne Praeprozessor.

\section{Debugging}
Um die Funktionsweise zu erläutern schleusen wir Debugging-Code in die Programme.
Es handelt sich dabei um die Ausgabe des Funktionsnamen, der Zeile und Kommentar.
Das Macro ist dargestellt in \Cref{code:debug}

\begin{code}
	\caption{Debugging Macro}
	\label{code:debug}
	\inputminted{c}{code/employees_virt/debug.h}
\end{code}


\section{Klassen}
Eine Klasse definiert einen Bauplan für einen neuen Datentypen.
Vor Allem legen wir fest welche Daten die Klasse beinhalten soll.
Desweiteren definieren wir die Operation, die mit dieser Klasse gebunden sind.


\section{Klassendeklaration}

Die Stukturen kapseln die Daten ein.
In unseren Beispiel speichern wir von einem Eingetellter den Vornamen, den Namen und den Job, der er ausübt.

In C werden die Strukturen nicht automatisch initialisiert.
Deswegen, nach dem wir eine Struktur erzeugen, werden die Membervariablen üblicherweise zufällige Werte beinhalten.
Es ist die Aufgabe der Programmierer die Objekte in einen Konsistenten zustand zu bringen.
Deswegen benötigt jede Klasse mindestens einen Konstruktor.
Abhängig von dem Funktionalität, kann eine Klasse mehrere Konstruktoren haben.

Desweiteren, ist es die Aufgabe des Programmieres sich um die Speicherverwaltung zu kümmern.
Insbesondere auf dem Heap allozierter Speicher muss nach der Verwendung wieder freigegben werden, um Speicherlecks zu vermeiden.
Für diesen Zwecke benötigen wir eine Destruktor.
Der Destruktor muss manuell aufgerufen werden, da C uns keine Hilfmittel für die Automatisierung zur Verfügung stellt, z.B. kein Garbage-Kollektor oder Aufruf vom Destruktor beim Verlassen des Gültigekeitsbereiches des Objektes.

Der Konstruktor, Destruktor und die Memberfuktionen lagern wir extern aus.
Da die Memberfunktionen nicht an eine Struktur gebunden werden können, ist es hilfreich sich eine Namenskonvetione zu überlegen, die die Zugehörigkeit anzeigt. 
In den Beispielen werden wir die Memberfunktionen stets nach dem folgenden Schema benennen.
\begin{lstlisting}
<Memberfunktionsname> = <Klassenname>_<Funktionsname>
\end{lstlisting}

\begin{code}
	\caption{Einfache Klassen: employees.h}
	\inputminted{C}{code/employees_plain/employee.h}
\end{code}



\subsection{Klassenfunktionen}
In C bietet es sich an die Implementierung der Funktionen in separate Dateien auszulagern.
Das ist aber nicht zwingend erforderlich, aber wir werden es in dieser Anleitung so machen.
Die Implementierung koennte wie folgt aussehen.

\begin{code}
	\caption{Einfache Klassen: employees.c}
	\inputminted{C}{code/employees_plain/employee.c}
\end{code}


\subsection{Nutzung}
Bei der Benutzung sollen wir beachten, dass wir den Konstruktor und den Destruktor manuell aufrufen. 
\begin{code}
	\caption{Einfache Klassen: main.c}
	\inputminted{C}{code/employees_plain/main.c}
\end{code}

\begin{code}
	\caption{Einfache Klassen: Ausgabe}
	\inputminted[bgcolor=white]{text}{code/employees_plain/output.txt}
\end{code}

\subsection{Kritik}
\subsubsection{Zugriffskontrolle}
C-Standard bietet und keine Konstrukte mit den mit den man den Zugriff auf die Klassenvariablen und Klassenfunktionen beschränken könnte. Alle Variablen und Methoden sind \mintinline{C}|public|. Nachbildung von \mintinline{C}|protected| \mintinline{C}|private| nicht ohne weiteres möglich.
\subsubsection{Konstruktor und Destruktor}
Es ist leider so, dass beim Erzeugen eines Objektes der passende Konstruktor nicht automatisch aufgerufen wird, wie z.B. bei C++ oder Java.
Darum muss sich der Programmier explizit kümmern.
Das erfordert zwar viel Disziplin, bedeutet für uns keine Nachteile in der Funktionalität.
Bei den Destruktoren müssen wir leider Abstriche machen.
C bietet uns weder eine Garbage-Kollektor noch Ruf den Destrutkor automatisch auf, wenn das Programm den Scope verlässt.
Konkret für uns heißt es wieder viel Disziplin, aber auch das wir bestimmte Programmierkonzepte wie "RAII" nicht umsetzen können.

\begin{note}
	Das geht mit der clean() Erweiterung. Das ist aber keine Standard-C, somit wird diese MÖglichkeit hier nicht behandelt.
\end{note}




Wie wir spaeter sehen werden liegt der Vorteil der einfachen Klassen im geringeren Speicherverbrauch durch den fehlenden Zeiger auf eine virtuelle Tabelle.

\section{Vererbung}
\subsection{C++ als Vorbild}
Dieser Abschnitt zeigt einen Beispiel einer Vererbung in C++.
In einem Unternehmen werden von den gewöhnlichen Angestellten nur der Vorname und Nachname gespeichert.
Es ist möglich den Vornamen und Namen ausgeben zu lassen.

\begin{code}
	\caption{C++ Beispiel: \mintinline{cpp}{Employee} Klasse}
	\label{code:virtcpp:employee}
	\inputminted{cpp}{code/examples/employees_virt_cpp/employee.hpp}
	%\inputminted{cpp}{code/employees_virt_cpp/employee.cpp}
\end{code}

Ein Manager ist ein Angestellter mit einem bestimmten Level.
Ihm kann eine Gruppe mit einer bestimmter Anzahl der Angestellten zugeordnet werden.
Es ist möglich alle diese Information über einen Manager anzeigen zu lassen.

\begin{code}
	\caption{C++ Beispiel: \mintinline{cpp}{Manager} Klasse}
	\label{code:virtcpp:manager}
	\inputminted{cpp}{code/examples/employees_virt_cpp/manager.hpp}
	%\inputminted{cpp}{code/employees_virt_cpp/manager.cpp}
\end{code}

Dieser Zusammenhang ist abgebildet durch die Klassen \mintinline{cpp}{Employee} und \mintinline{cpp}{Manager}.

Eine mögliche Nutzung ist in \Cref{code:virtcpp:usage} dargestellt.
Wir haben in den Konstruktor, Destruktor und Memberfunktionen Debugging-Code eingeschleusst um die Aufrufreihenfolge zu verfolgen.
Als Kommentar zur jeder Ausgabe steht auch der Vorname und der Nachname der Angestellten, um die Aufrufe zuzuordnen zu können.
In der Ausgabe kann man sehen, dass beim Erzeugen des Objektes stets ein passender Konstruktor aufgerufen wird.
Am Ende werden die Objekte nach dem LIFO Prizip automatisch zerstört, weil der Gültigkeitsbereich der Objekte, der sich auf die \mintinline{cpp}{main} Funktion beschränkt, verlassen wird.
Zwischen den Konstruktor- und Destruktoraufrufen sehen wir die Aufrufe der Memberfunktionen und die Ausgabe von \mintinline{cpp}{printf}.

\begin{code}
	\caption{C++ Beispiel: Nutzung}
	\label{code:virtcpp:usage}
	\inputminted{cpp}{code/examples/employees_virt_cpp/main.cpp}
\end{code}

\begin{code}
	\caption{C++ Beispiel: Ausgabe}
	\label{code:virtcpp:output}
	\inputminted[bgcolor=white]{text}{code/examples/employees_virt_cpp/output.txt}
\end{code}

Dieses Verhalten wollten wir in C nachbilden.

\input{03_2_concepts}
\input{03_3_classes_virt_c}
%\tikzset{
% treenode/.style = {align=center, inner sep=4pt, text centered, font=\sffamily},
% arn_n/.style = {treenode, rectangle, font=\sffamily}
%}

%\begin{figure}[h]
%  \centering
%\begin{tikzpicture}[<-,>=stealth',level/.style={sibling distance = 5.5cm/#1,
% level distance = 1cm}] 
% \node [arn_n] {\mintinline{C}{shape_t}}
%   child{ node [arn_n] {\mintinline{C}{line_t}} 
%     child{ node [arn_n] {\mintinline{C}{arrow_t}}}
%   }
%   child{ node [arn_n] {\mintinline{C}{cirle_t}}
%     child{ node [arn_n] {\mintinline{C}{arc_t}}}
%     child{ node [arn_n] {\mintinline{C}{angle_t}}}
%   }
%; 
%\end{tikzpicture}
%\caption{Klassen Hierarchie}
%\label{fig:hierarchy}
%\end{figure}


\inputminted{c}{code/examples/employees_virt/main2_short.c}
\inputminted{text}{code/examples/employees_virt/output2.txt}
		%\begin{itemize}
			%\item Objekte mit der gleicher Basis können z.B. in einem Array gespeichert werden
			%\item Es wird die richtige virtuelle Funktion aufgerufen
		%\end{itemize}




\section{Abstrakte Klassen}

TODO

%%https://www.embedded.com/electronics-blogs/programming-pointers/4391967/Virtual-functions-in-C
%%http://vgcoding.blogspot.com/2014/03/virtual-functions-in-c.html

%\begin{frame}[fragile]{Konzept der virtuellen Funktionen}
%  %Quelle: \cite{virtfunc2012}
%\end{frame}

%%\section{Grenzen}
%%\begin{frame}[fragile]{Eigenschaften}
%  %\begin{itemize}
%    %\item Konvetionen
%      %\begin{itemize}
%        %\item Funktionensname haben den Praefix der Struktur
%      %\end{itemize}
%  %\end{itemize}
%%\end{frame}





%%\section{Zusammenfassung}
%%\subsection*{}

%%\begin{frame}
%%  \frametitle{Zusammenfassung}
%%\end{frame}

%\section{Literatur}
%\subsection*{}

%\begin{frame}
%  %\frametitle{Literatur}
%  %\printbibliography
%\end{frame}

\end{document}
