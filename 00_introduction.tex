\section{Einleitung}
Obwohl C keine objektorientierte Programmierung unterstützt, kann man mit der Sprache den Konzept der objektorientierten Programmierung umsetzen.
Das erfordert allerdings ein sehr viel Disziplin, weil C-Sprache den Programmierer kaum dabei unterstützt.
Sie eschwert es sogar, indem sie viele Umsetzungsmöglichkeiten bietet.
In dieser Ausarbeitung behandeln nur eine Möglichkeit von Vielen objektorientiert zu Programmieren.
Unsere Absicht ist möglichst nicht über den C-Standard hinausgehen, d.h. ohne Verwendung von exotischen Kompilererweiterungen und mit minimalen Einsatz von Präprozessor.


\subsection{Debugging}
Um die Funktionsweise zu erläutern schleusen wir Debugging-Code in die Programme.
Es handelt sich dabei um die Ausgabe des Funktionsnamen, der Zeile und Kommentar.
Das Macro ist dargestellt in \Cref{code:debug}

\begin{code}
	\caption{Debugging Macro}
	\label{code:debug}
	\inputminted{c}{code/examples/employees_virt/debug.h}
\end{code}

\subsection{Kompilierung}
Die anonymen Strukturen haben sich als esseziell für die virtuellen Strukturen erwiesen.
Um sie zu aktivieren, verwenden wir bei der Kompilierung den folgenden Kompiler-Flag.
\begin{code}
	\begin{minted}{bash}
gcc -fms-extensions
	\end{minted}
\end{code}
