\subsection{Nameskonflikte}

Bevor wir die Vererbung implementieren, müssen wir es die Entscheidung treffen wie wir mit den Namenkonflikten umgehen.
Wir haben die Wahl zwischen einer benannten oder anonymen Basisklasse.
Bei der ersteren ist der Zugriff auf die gleichbenannte Variable in der Basisklasse immer gewährleistet.
Allerdings muss der Zugriff immer über die Basisklasse gehen.
Bei der zweiten MÖglichkeit erlaubt uns der C-Standard keine anonyme Struktur, wenn eine gleichnamige Variable bereits in der Basisklasse vorhanden ist.
Das schränkt zwar in der Wahl der Variablenname etwas ein, aber der Zugriff auf die Variablen in der Basisklasse wird kürzer.
Wir werden die erste Möglichkeit verwenden, da sie uns weniger einschränkt.

\noindent\begin{minipage}{.45\textwidth}
\begin{minted}{C}
struct A {
 int p;
};

struct B {
 A super;
 int q;
 int p; // Kein Namenskonflikt
};

B b;
b.super.p = 5;
b.q = 6;
b.p = 7;
\end{minted}
\end{minipage}\hfill
\begin{minipage}{.45\textwidth}
\begin{minted}{C}
struct A {
 int p;
};

struct B {
 A;
 int q;
 // int p; 
 // Name bereits vergeben
};

B b;
b.p = 5;
b.q = 6;
\end{minted}
\end{minipage}







%\begin{frame}[fragile]{Beispiel}
%\begin{columns}
%\column{0.65\textwidth}


\tikzset{
 treenode/.style = {align=center, inner sep=4pt, text centered, font=\sffamily},
 arn_n/.style = {treenode, rectangle, font=\sffamily}
}


\begin{figure}[h]
	\centering
\begin{tikzpicture}[<-,>=stealth',level/.style={sibling distance = 5.5cm/#1,
 level distance = 1cm}] 
 \node [arn_n] {\mintinline{C}{shape_t}}
	 child{ node [arn_n] {\mintinline{C}{line_t}} 
		 child{ node [arn_n] {\mintinline{C}{arrow_t}}}
	 }
	 child{ node [arn_n] {\mintinline{C}{cirle_t}}
		 child{ node [arn_n] {\mintinline{C}{arc_t}}}
		 child{ node [arn_n] {\mintinline{C}{angle_t}}}
	 }
; 
\end{tikzpicture}
\caption{Klassen Hierarchie}
\label{fig:hierarchy}
\end{figure}

%  \begin{figure}
%    \begin{center}
%      \includegraphics[width=0.9\textwidth]{type_sizes.png}
%    \end{center}
%    \caption{Beispielgrafik}
%  \end{figure}

\inputminted{C}{code/employees/employee.h}


%\end{frame}


%\begin{frame}[fragile]{Anonyme Strukturen}

\begin{code}
\caption{Basisklasse}
\begin{minted}{C}
struct base_t { int a; };
\end{minted}
\end{code}

\begin{code}
\caption{Anonyme Struktur}
\begin{minted}{C}
struct derived_t {
 struct base_t;
 int b;
};
\end{minted}
\end{code}

\begin{code}
\caption{Zugriff}
\begin{minted}{C}
struct derived_t d;
d.a = 5;
d.b = 6;
\end{minted}
\end{code}

\begin{code}
\caption{Beispielquelltext}
\begin{minted}{C}
struct derived_t {
 struct base_t this;
 int b;
};
\end{minted}
\end{code}

\begin{code}
\caption{Beispielquelltext}
\begin{minted}{C}
struct derived_t d;
d.this.a = 5;
d.b = 6;
\end{minted}
\end{code}
