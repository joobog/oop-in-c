%\tikzset{
% treenode/.style = {align=center, inner sep=4pt, text centered, font=\sffamily},
% arn_n/.style = {treenode, rectangle, font=\sffamily}
%}

%\begin{figure}[h]
%  \centering
%\begin{tikzpicture}[<-,>=stealth',level/.style={sibling distance = 5.5cm/#1,
% level distance = 1cm}] 
% \node [arn_n] {\mintinline{C}{shape_t}}
%   child{ node [arn_n] {\mintinline{C}{line_t}} 
%     child{ node [arn_n] {\mintinline{C}{arrow_t}}}
%   }
%   child{ node [arn_n] {\mintinline{C}{cirle_t}}
%     child{ node [arn_n] {\mintinline{C}{arc_t}}}
%     child{ node [arn_n] {\mintinline{C}{angle_t}}}
%   }
%; 
%\end{tikzpicture}
%\caption{Klassen Hierarchie}
%\label{fig:hierarchy}
%\end{figure}


\inputminted{c}{code/examples/employees_virt/main2_short.c}
\inputminted{text}{code/examples/employees_virt/output2.txt}
		%\begin{itemize}
			%\item Objekte mit der gleicher Basis können z.B. in einem Array gespeichert werden
			%\item Es wird die richtige virtuelle Funktion aufgerufen
		%\end{itemize}
